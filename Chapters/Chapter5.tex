% Chapter Template

\chapter{Conclusiones} % Main chapter title

\label{Chapter5} % Change X to a consecutive number; for referencing this chapter elsewhere, use \ref{ChapterX}


%----------------------------------------------------------------------------------------

%----------------------------------------------------------------------------------------
%	SECTION 1
%----------------------------------------------------------------------------------------

En este capítulo se presenta las conclusiones del trabajo en general, los principales aportes obtenidos y los próximos pasos.

\section{Conclusiones generales }

El presente trabajo cumplió con todos los requerimientos planteados en la sección \ref{requerimientos}, con la salvedad del requerimiento opcional de documentación, donde se planteó documentar en un archivo README.md dentro de un repositorio. Este último requerimiento opcional, no se realizó por falta de tiempo.

En cuanto al cronograma no se cumplió fielmente como se planteó durante la planificación, ya que, algunas de las tareas se ejecutaron de forma más rápida que otras y hubo otras que duraron más del tiempo planificado por su complejidad y que requerían mayor dedicación. Las tareas que demandaron mayor cantidad de tiempo y dedicación son las siguientes:

\begin{itemize}
\item Etiquetado de imágenes.
\item Redacción de la memoria final del proyecto. 
\end{itemize}

Por otro lado, a pesar de cumplir con todos los requerimientos funcionales se tuvo una tarea que no se finalizo en su totalidad y otras que no se realizaron. Con esto, la tarea que no se logró finalizar por falta de tiempo fue el etiquetado de todas la imágenes del conjunto de datos y la tarea que no se realizó fue la de desarrollar un \textit{endpoint}, debido a que no se requirió el uso de dicho \textit{endpoint} al ser un despliegue local. Por lo tanto, al no tener que disponibilizar un endpoint tampoco se realizó la tarea de hacer pruebas de funcionamiento sobre ese \textit{endpoint}.



%La idea de esta sección es resaltar cuáles son los principales aportes del trabajo realizado y cómo se podría continuar. Debe ser especialmente breve y concisa. Es buena idea usar un listado para enumerar los logros obtenidos.
%
%Algunas preguntas que pueden servir para completar este capítulo:
%
%\begin{itemize}
%\item ¿Cuál es el grado de cumplimiento de los requerimientos?
%\item ¿Cuán fielmente se puedo seguir la planificación original (cronograma incluido)?
%\item ¿Se manifestó algunos de los riesgos identificados en la planificación? ¿Fue efectivo el plan de mitigación? ¿Se debió aplicar alguna otra acción no contemplada previamente?
%\item Si se debieron hacer modificaciones a lo planificado ¿Cuáles fueron las causas y los efectos?
%\item ¿Qué técnicas resultaron útiles para el desarrollo del proyecto y cuáles no tanto?
%\end{itemize}


%----------------------------------------------------------------------------------------
%	SECTION 2
%----------------------------------------------------------------------------------------
\section{Próximos pasos}

Acá se indica cómo se podría continuar el trabajo más adelante.
