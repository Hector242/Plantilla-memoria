% Chapter Template

\chapter{Conclusiones} % Main chapter title

\label{Chapter5} % Change X to a consecutive number; for referencing this chapter elsewhere, use \ref{ChapterX}


%----------------------------------------------------------------------------------------

%----------------------------------------------------------------------------------------
%	SECTION 1
%----------------------------------------------------------------------------------------

En este capítulo se presenta las conclusiones del trabajo en general, los principales aportes obtenidos y los próximos pasos.

\section{Conclusiones generales }

El presente trabajo cumplió con todos los requerimientos planteados en la sección \ref{requerimientos}, con la salvedad del requerimiento opcional de documentación, donde se planteó documentar en un archivo README.md dentro de un repositorio. Este último requerimiento opcional, no se realizó por falta de tiempo.

En cuanto al cronograma no se cumplió fielmente como se planteó durante la planificación, ya que, algunas de las tareas se ejecutaron de forma más rápida que otras y hubo otras que duraron más del tiempo planificado por su complejidad y que requerían mayor dedicación. Las tareas que demandaron mayor cantidad de tiempo y dedicación son las siguientes:

\begin{itemize}
\item Etiquetado de imágenes.
\item Redacción de la memoria final del proyecto. 
\end{itemize}

Por otro lado, a pesar de cumplir con todos los requerimientos funcionales, se tuvo una tarea que no se finalizó en su totalidad y otras dos que no se realizaron. Con esto, la tarea que no se logró finalizar por falta de tiempo fue el etiquetado de todas la imágenes del conjunto de datos y la tarea que no se realizó, fue la de desarrollar un \textit{endpoint} para el consumo del sistema, esta no se desarrolló debido a que no se requirió el uso de dicho \textit{endpoint} al ser un despliegue local. Por lo tanto, al no tener que disponibilizar un endpoint tampoco se realizó la tarea de hacer pruebas de funcionamiento sobre ese \textit{endpoint}.

Durante la ejecución del presente trabajo se manifestaron los siguientes riesgos descritos en la planificación:

\begin{itemize}
\item Datos insuficientes. La cantidad de imágenes proporcionadas es insuficiente o su calidad no es la adecuada para entrenar el modelo.
\item Etiquetar incorrectamente los estados fenológicos de la flor. 
\end{itemize}

Ambos riesgos fueron mitigados siguiendo plan de mitigación diseñado durante la planificación. Donde el primer riesgo requirió el uso de \textit{transfer learning} y aumento de datos, y para el segundo riesgo se requirió la participación del especialista del INTA para identificar correctamente los estados fenológicos en las fotos proporcionadas.

Por último, se destacan los logros del presente trabajo:

\begin{itemize}
\item Se desarrolló un algoritmo capaz de identificar los estados fenológicos de la flor de duraznero y extraer información de su vareta (cantidad de flores, tipo de flor y longitud en centímetros) a partir de imágenes.
\item Se diseño el sistema para funcionar en una computadora local sin la necesidad de GPU y con la posibilidad de funcionar en distintos sistemas operativos.
\item Se realizó una interfaz de usuario para su uso y acceso.
\item El sistema acepta distintos formatos de imágenes de entrada entre los cuales esta \textit{JPG}, \textit{JPEG}, \textit{PNG}.
\item Los resultados se entregan en formato \textit{CSV} y \textit{XLSX} para \textit{Excel}.
\end{itemize} 

%La idea de esta sección es resaltar cuáles son los principales aportes del trabajo realizado y cómo se podría continuar. Debe ser especialmente breve y concisa. Es buena idea usar un listado para enumerar los logros obtenidos.
%
%Algunas preguntas que pueden servir para completar este capítulo:
%
%\begin{itemize}
%\item ¿Cuál es el grado de cumplimiento de los requerimientos?
%\item ¿Cuán fielmente se puedo seguir la planificación original (cronograma incluido)?
%\item ¿Se manifestó algunos de los riesgos identificados en la planificación? ¿Fue efectivo el plan de mitigación? ¿Se debió aplicar alguna otra acción no contemplada previamente?
%\item Si se debieron hacer modificaciones a lo planificado ¿Cuáles fueron las causas y los efectos?
%\item ¿Qué técnicas resultaron útiles para el desarrollo del proyecto y cuáles no tanto?
%\end{itemize}


%----------------------------------------------------------------------------------------
%	SECTION 2
%----------------------------------------------------------------------------------------
\section{Próximos pasos}

Acá se indica cómo se podría continuar el trabajo más adelante.
