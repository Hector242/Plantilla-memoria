\chapter{Diseño e implementación} % Main chapter title

\label{Chapter3} % Change X to a consecutive number; for referencing this chapter elsewhere, use \ref{ChapterX}

En este capítulo se describe el diseño e implementación del sistema desarrollado que automatiza la toma de datos de las flores de duraznero a través de fotos de varetas.

\section{Arquitectura del sistema}

El sistema general implementado en este trabajo se presenta en la figura \ref{fig:sistemaGeneral}. Este sistema cuenta con un \textit{frontend} donde el usuario sube una foto de vareta de duraznero y posteriormente esta imagen pasará por dos módulos de forma secuencial. El primer modulo, es el encargado de medir la longitud de la vareta y el segundo modulo se encarga de detectar el estado fenológico, la cantidad de flores y el tipo de flores que posee la vareta. Cada uno de los resultados obtenidos por estos modulos es presentado por pantalla al usuario y con la posibilidad de descargarlos en formato CSV.

\begin{figure}[ht]
	\centering
	\includegraphics[scale=.4]{./Figures/arq1.drawio.png}
	\caption{Arquitectura general del sistema.}
	\label{fig:sistemaGeneral}
\end{figure}

La arquitectura del modulo que estima la longitud de la vareta se puede observar en la figura \ref{fig:varetaSize}. Este modulo toma y preprocesa la imagen de entrada, posteriormente detecta la regla que es el objeto de referencia y procede a tomar las mediciones en píxeles de dicho elemento. Luego, se hace una conversión de píxeles a centímetros. Una vez finalizada la conversión, se detecta las varetas presentes en la imagen y se calculan sus dimensiones, se revisa la orientación de la foto y se toma la longitud de cada vareta en píxeles. Finalmente, con la conversión anterior se pasa a centímetros las mediciones de las varetas y se anotan los resultados en una tabla y son enviados al siguiente modulo. Por otro lado, la imagen con las mediciones se muestra por pantalla.

Este modulo se detalla más a profundidad en este mismo capítulo.

\begin{figure}[htpb]
	\centering
	\includegraphics[scale=.4]{./Figures/medicionVareta.drawio.png}
	\caption{Modulo de estimación de longitud.}
	\label{fig:varetaSize}
\end{figure}

La arquitectura del modulo de detección y procesamiento se presenta en la figura \ref{fig:densidadDeFlores}. La entrada a este modulo es la imagen original donde se le aplica un preprocesamiento y posteriormente se pasa por un modelo que detecta los estados fenológicos de las flores de duraznero y sus varetas. Luego, se realiza un postprocesamiento de los resultados para conocer el número de flores totales que se encuentran en la imagen y un conteo de las mismas por vareta. Por último, se extraen las flores detectadas por vareta y son enviadas al clasificador para determinar su tipo.

Más adelante en este mismo capítulo se dará más información detallada de este modulo y de otros bloques del sistema que cumplen una función importante para generar los resultados deseados.

Cabe destacar que la realización de este sistema se llevo a cabo en el lenguaje de programación \textit{Python} y esta diseñado para funcionar en una computadora local bajo un ambiente virtualizado. 

\begin{figure}[htp]
	\centering
	\includegraphics[scale=.35]{./Figures/DetecciónFlor.drawio.png}
	\caption{Modulo de detección y procesamiento.}
	\label{fig:densidadDeFlores}
\end{figure}

\section{Preparación de los datos}

%\definecolor{mygreen}{rgb}{0,0.6,0}
%\definecolor{mygray}{rgb}{0.5,0.5,0.5}
%\definecolor{mymauve}{rgb}{0.58,0,0.82}
%
%%%%%%%%%%%%%%%%%%%%%%%%%%%%%%%%%%%%%%%%%%%%%%%%%%%%%%%%%%%%%%%%%%%%%%%%%%%%%%
%% parámetros para configurar el formato del código en los entornos lstlisting
%%%%%%%%%%%%%%%%%%%%%%%%%%%%%%%%%%%%%%%%%%%%%%%%%%%%%%%%%%%%%%%%%%%%%%%%%%%%%%
%\lstset{ %
%  backgroundcolor=\color{white},   % choose the background color; you must add \usepackage{color} or \usepackage{xcolor}
%  basicstyle=\footnotesize,        % the size of the fonts that are used for the code
%  breakatwhitespace=false,         % sets if automatic breaks should only happen at whitespace
%  breaklines=true,                 % sets automatic line breaking
%  captionpos=b,                    % sets the caption-position to bottom
%  commentstyle=\color{mygreen},    % comment style
%  deletekeywords={...},            % if you want to delete keywords from the given language
%  %escapeinside={\%*}{*)},          % if you want to add LaTeX within your code
%  %extendedchars=true,              % lets you use non-ASCII characters; for 8-bits encodings only, does not work with UTF-8
%  %frame=single,	                % adds a frame around the code
%  keepspaces=true,                 % keeps spaces in text, useful for keeping indentation of code (possibly needs columns=flexible)
%  keywordstyle=\color{blue},       % keyword style
%  language=[ANSI]C,                % the language of the code
%  %otherkeywords={*,...},           % if you want to add more keywords to the set
%  numbers=left,                    % where to put the line-numbers; possible values are (none, left, right)
%  numbersep=5pt,                   % how far the line-numbers are from the code
%  numberstyle=\tiny\color{mygray}, % the style that is used for the line-numbers
%  rulecolor=\color{black},         % if not set, the frame-color may be changed on line-breaks within not-black text (e.g. comments (green here))
%  showspaces=false,                % show spaces everywhere adding particular underscores; it overrides 'showstringspaces'
%  showstringspaces=false,          % underline spaces within strings only
%  showtabs=false,                  % show tabs within strings adding particular underscores
%  stepnumber=1,                    % the step between two line-numbers. If it's 1, each line will be numbered
%  stringstyle=\color{mymauve},     % string literal style
%  tabsize=2,	                   % sets default tabsize to 2 spaces
%  title=\lstname,                  % show the filename of files included with \lstinputlisting; also try caption instead of title
%  morecomment=[s]{/*}{*/}
%}
%
%
%%----------------------------------------------------------------------------------------
%%	SECTION 1
%%----------------------------------------------------------------------------------------
%\section{Análisis del software}
% 
%La idea de esta sección es resaltar los problemas encontrados, los criterios utilizados y la justificación de las decisiones que se hayan tomado.
%
%Se puede agregar código o pseudocódigo dentro de un entorno lstlisting con el siguiente código:
%
%\begin{verbatim}
%\begin{lstlisting}[caption= "un epígrafe descriptivo"]
%	las líneas de código irían aquí...
%\end{lstlisting}
%\end{verbatim}
%
%A modo de ejemplo:
%
%\begin{lstlisting}[label=cod:vControl,caption=Pseudocódigo del lazo principal de control.]  % Start your code-block
%
%#define MAX_SENSOR_NUMBER 3
%#define MAX_ALARM_NUMBER  6
%#define MAX_ACTUATOR_NUMBER 6
%
%uint32_t sensorValue[MAX_SENSOR_NUMBER];		
%FunctionalState alarmControl[MAX_ALARM_NUMBER];	//ENABLE or DISABLE
%state_t alarmState[MAX_ALARM_NUMBER];						//ON or OFF
%state_t actuatorState[MAX_ACTUATOR_NUMBER];			//ON or OFF
%
%void vControl() {
%
%	initGlobalVariables();
%	
%	period = 500 ms;
%		
%	while(1) {
%
%		ticks = xTaskGetTickCount();
%		
%		updateSensors();
%		
%		updateAlarms();
%		
%		controlActuators();
%		
%		vTaskDelayUntil(&ticks, period);
%	}
%}
%\end{lstlisting}



