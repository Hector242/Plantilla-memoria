% Chapter Template

\chapter{Ensayos y resultados} % Main chapter title

\label{Chapter4} % Change X to a consecutive number; for referencing this chapter elsewhere, use \ref{ChapterX}

%----------------------------------------------------------------------------------------
%	SECTION 1
%----------------------------------------------------------------------------------------
En este capítulo se describen los ensayos realizados y se presentan y comparan los resultados obtenidos.

\section{Banco de pruebas}
\label{sec:pruebasHW}

Para el desarrollo del presente proyecto se utilizaron 3 bancos de pruebas, con las siguientes especificaciones técnicas:

\begin{itemize}
\item Banco de pruebas \textit{Laptop} \textit{MacBook Pro} - \textit{hardware} disponible:
	\begin{itemize}
	\item \textbf{OS}: \textit{MacOS Sonoma} 14.
	\item \textbf{CPU}: Apple M1 Pro.
	\item \textbf{RAM}: 32 GB.
	\item \textbf{GPU}: No contiene.
	\end{itemize}
\item Banco de pruebas \textit{Laptop} \textit{ASUS TUF GAMING F15} - \textit{hardware} disponible:
	\begin{itemize}
	\item \textbf{OS}: \textit{Windows} 11.
	\item \textbf{CPU}: \textit{Intel(R) Core(TM)} i5-11260H.
	\item \textbf{RAM}: 16 GB.
	\item \textbf{GPU}: \textit{NVIDIA GeForce RTX} 3050.
	\end{itemize}
\item Banco de pruebas \textit{Google Colab Pro} - \textit{hardware} disponible:
	\begin{itemize}
	\item \textbf{OS}: desconocido.
	\item \textbf{CPU}: desconocido.
	\item \textbf{RAM}: 12,5 GB.
	\item \textbf{GPU}: T4.
	\end{itemize}
\end{itemize}

\section{Desempeño de modelos}
\label{sec:desempeñoMod}

Para medir el desempeño de los modelos de AI utilizados durante la ejecución de este proyecto, se decidió emplear la métrica \textit{mAP}. Esta métrica evalúa tanto la clasificación del objeto como la ubicación del \textit{bounding box} a través del \textit{Avarage Precision} (\textit{AP}) que se representa en la ecuación \ref{eq:AP}.

\begin{equation}
    \label{eq:AP}
    \text{AP} = \frac{1}{n} \sum_{k=1}^n \text{Precisión en } k \times \text{Rel}_{k}
\end{equation}

Siendo
\begin{itemize}
    \item \( n \) el número total de elementos recuperados.
    \item \(\text{Precision at } k\) la precisión en el \( k \)-ésimo punto de recuperación.
    \item \(\text{Rel}_{k}\) un indicador binario que denota si el elemento en el \( k \)-ésimo punto de recuperación es relevante (\( \text{Rel}_{k} = 1 \)) o no relevante (\( \text{Rel}_{k} = 0 \)).
\end{itemize}

Luego, al promediar los valores de \textit{AP} entre todas las clases, se obtiene respectivamente el \textit{mAP} y la ecuación quedaría como se muestra en \ref{eq:mAP}.

\begin{equation}
    \label{eq:mAP}
    \text{mAP} = \frac{1}{C} \sum_{c=1}^C \text{AP}_c
\end{equation}

Donde 
\begin{itemize}
	\item \( C \) es el número total de clases o consultas.
    \item \( AP \) es el Average Precision calculado para la clase o consulta.
\end{itemize}

\subsection{Desempeño del detector de regla}

El detector de la regla como se mencionó en el capítulo anterior, cumple una función fundamental para la estimación de la longitud de la vareta. Por otro lado, el entrenamiento de este modelo se realizó con los siguientes hiperparámetros:

\begin{itemize}
	\item Epocas: 100.
    \item Tamaño de imagen de entrada: 640 x 640.
    \item \textit{Batch}: 16.
    \item \textit{Learning rate}: 0.01.
\end{itemize}

Los resultados obtenidos para la detección de este elemento contra el conjunto de datos de prueba se observan en la tabla \ref{tab:resultadosRegla}.

\begin{table}[h]
	\centering
	\caption{Métricas de detección para el detector de regla.}
	\begin{tabular}{c c c c c c c}    
		\toprule
		\textbf{Clase}&\textbf{Imágenes}&\textbf{Muestras}&\textbf{Precision} &\textbf{Recall}&\textbf{mAP 50}&\textbf{mAP 50-95}\\
		\midrule
		Todas & 13 & 13 & 0.996 & 1.0 & 0.995 & 0.995\\		
		\bottomrule
		\hline
	\end{tabular}
	\label{tab:resultadosRegla}
\end{table}

Los resultados fueron muy buenos en general. El modelo puede detectar la regla sin dificultad en el 99\% de los casos.

\subsection{Desempeño del detector de vareta}

\section{Desempeño de módulos}
\label{sec:desempeñoModulos}